\documentclass[11pt]{article}

\usepackage{../theory}

\begin{document}
\date{}
\coverpage{5}
\maketitle

\newpage
\section*{Problem 1}

\noindent
We are given 5 matrices $M_1,...,M_5$, their dimensions (i.e., rows by columns)
are as follows: 
$M_1$ is 15 $\times$ 20,
$M_2$ is 20 $\times$ 30,
$M_3$ is 30 $\times$ 10,
$M_4$ is 10 $\times$ 50, and
$M_5$ is 50 $\times$ 8. \newline

\noindent
(1.1) Run the dynamic programming algorithm for {\em matrix chain multiplication} that we covered in class to produce the table $m[-,-]$. \parspace
We give the table $m[-,-]$:
\begin{center}
	\begin{tabular}{|c|c|c|c|c|c|}
		\hline
		& 1 & 2 & 3 & 4 & 5 \\
		\hline
		1 & 0 & 9000 & 9000 & 21000 & 13600 \\
		\hline
		2 & -- & 0 & 6000 & 16000 & 11200 \\
		\hline
		3 & -- & -- & 0 & 15000 & 6400 \\
		\hline
		4 & -- & -- & -- & 0 & 4000 \\
		\hline
		5 & -- & -- & -- & -- & 0 \\
		\hline
	\end{tabular}
\end{center}

\noindent
(1.2) What is the optimal solution value? Where do you find it? \parspace
The optimal solution value is 136000 multiplications. This value can be found in the top right corner of the table at $m[1,5]$.

\newpage
\section*{Problem 2}

\noindent
We are given a context-free grammar $G$ as follows:
\newline

$G$: $S \longrightarrow AS \mid SB$

~~~~  $A \longrightarrow AD\mid DA \mid a$

~~~~  $B \longrightarrow BB \mid BD \mid b$

~~~~  $D \longrightarrow DD \mid d$

We are also given a string $w=bdbdd$.
\newline

\noindent
(2.1) Run the dynamic programming algorithm for $A_{CFG}$ that we covered in class to produce the table $table[-,-]$. \parspace
We give now fill the entries of $table[-,-]$:
\begin{center}
	\begin{tabular}{|c|c|c|c|c|c|}
		\hline
		& 1 & 2 & 3 & 4 & 5 \\
		\hline
		1 & B & B & B & B & B \\
		\hline
		2 & -- & D & $\emptyset$ & $\emptyset$ & $\emptyset$ \\
		\hline
		3 & -- & -- & B & B & B \\
		\hline
		4 & -- & -- & -- & D & D \\
		\hline
		5 & -- & -- & -- & -- & D \\
		\hline
	\end{tabular}
\end{center}

\noindent
(2.2) How do we know whether $G$ generates $w$ from the table? \parspace
We know whether $G$ generates $w$ by the following rule:
$$w \in L(G) \iff S \in table[1,5]$$
Since $S \notin table[1,5]$ we know that $G$ does not generate $w$.

\newpage
\section*{Problem 3}

Show that $ALL_{DFA} \in $ P. \parspace
We first define $ALL_{DFA} := \{ \langle D \rangle \mid D$ is a DFA and $L(D) = \Sigma ^* \}$. \parspace
From Theorem 4.4, we know that $E_{DFA} := \{ \langle A \rangle \mid A $ is a DFA and $L(A) = \emptyset \}$ is decidable by the following Turing Machine. \parspace
$T = $ ``On input $\langle A \rangle$ a DFA: \\
\indent \textbf{1. } Mark the start state of $A$ \\
\indent \textbf{2. } Repeat until no new states get marked: \\
\indent \textbf{3. } \indent Mark any state that has a transition coming into it from any state that 
\indent \indent \indent ~is already marked \\
\indent \textbf{4. } If no accept state is marked, \textit{accept}; otherwise, \textit{reject}
'' \parspace
Additionally, we can construct the complement of a DFA by reversing accept/nonaccept states in the DFA. We now give a Turing Machine $A$ to decide $ALL_{DFA}$. \parspace
$A = $ ``On input $\langle D \rangle$ a DFA: \\
\indent \textbf{1. } Construct a DFA $E$ where $L(E) = \overline{L(D)}$ \\
\indent \textbf{2. } Run TM $T$ on input $\langle E \rangle$ \\
\indent \textbf{3. } If $T$ accepts, \textit{accept} \\
\indent \textbf{4. } If $T$ rejects, \textit{reject}
'' \parspace
In words, $A$ constructs the DFA that is the complement of $D$. It then tests if the complement's language is empty. If so, the original language must be $\Sigma ^*$. \parspace
So $A$ certainly decides $ALL_{DFA}$, but does it do so in polynomial time? Let $s$ be the number of states for the input DFA $D$, then the number of transitions in $D$ is $| \Sigma | * s$. Line 1 of $A$ runs in $O(s)$ time. Lines 3 and 4 run in $O(1)$ time. The running time of line 2 is the same as the running time of $T$. \parspace
Lines 1 and 4 of $T$ run in $O(1)$ time while line 3 runs in $O(| \Sigma | * s)$ time (the number of edges). But $ | \Sigma | $ is a constant so $ O(| \Sigma | * s) = O(s)$. In the worst case scenario, line 2 repeats $s$ times so $T$ is a $O(1) + s * O(s) + O(1) = O(s^2)$ machine. \parspace
This brings us back to machine $A$, which we now know to be a $ O(1) + O(s^2) + O(1) + O(1) = O(s^2)$ machine. Since $ALL_{DFA}$ has a polynomial time decider, it must be true that $ALL_{DFA} \in P$. 

\newpage
\section*{Problem 4}

Show that Independent Set $\in$ NP. \parspace
We first define the problem: \parspace 
$ IS := \{ \langle G, k \rangle \mid G=(V,E) \text{ is a graph and } k \leq |V| \text{ where there exists an independent set }$ \\
\indent \indent \indent \indent \indent
$ V^\prime \subset V \text{ with } |V^\prime| \geq k \} $ \parspace
We must also define an independent set of vertices. This is a set $V^\prime \subset V$ such that for all $u,v \in V^\prime$ it is true that $(u, v) \notin E$. \parspace 
Note that a language $A$ is a member of NP if $A$ has a polynomial verifier. This is true if and only if $A$ is decidable by some nondeterministic Turing Machine in polynomial time. We now give a nondeterministic Turing Machine $S$ to decide $IS$. \parspace
$S =$ ``On input $\langle G, k \rangle$ where $G = (V, E)$ a graph and $k \in \mathbb{N}$: \\
\indent \textbf{1. } Nondeterministically select a subset of vertices $ V^\prime$ such that $| V^\prime | \geq k$ \\
\indent \textbf{2. } For each element $(u, v) \in V' \times V'$ \\
\indent \textbf{3. } \indent If $(u, v) \in E$, \textit{reject} \\
\indent \textbf{4. } Test has been passed, \textit{accept}
'' \parspace
Certainly $S$ nondeterminstically decides $IS$, but does it run in polynomial time? Yes, in fact, it runs in $ O(n^2) $ time, where $n = | V' |$. So we have a polynomial, nondeterminstic decider for $IS$, therefore $ S \in NP$.

\end{document}
