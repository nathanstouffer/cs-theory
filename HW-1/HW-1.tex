\documentclass[11pt]{article}

\usepackage{../theory}

\begin{document}

\coverpage{1}

\newpage
\section*{Problem 1}
Prove that $1^2+3^2+5^2+\cdots+(2n-1)^2=\frac{1}{3}n(4n^2-1)$.
\newline

\noindent
{\bf Proof: }
	We begin by defining the property 
	$P(n) := 1^2 + 3^2 + 5^2 + \cdots + (2n-1)^2 $ 
	for any 
	$n \in \mathbb{Z}^+$.
	Our task is to show that $P(n) = \frac{1}{3}n(4n^2-1)$ for all $n \in \zpos$. We proceed by induction. \parspace
	We first must prove the base case of $n = 1$. 
	Take the LHS: $P(n) = 1^2 = 1$. 
	And now the RHS: 
	$ \frac{1}{3}n(4n^2-1) = \frac{1}{3}(1)(4(1)^2-1) = 1$.
	Since the LHS and the RHS for both 1 for $n = 1$, our base case is proved. \parspace
	We now state the inductive hypothesis. That is, for some $k \in \zpos$, we assume
	$P(k) = \frac{1}{3}k(4k^2-1)$
	to be true. \parspace
	To complete our proof, we must now show that 
	$P(k+1) = \frac{1}{3}(k+1)(4(k+1)^2-1)$
	holds. First note that 
	$$P(k+1) = 1^2 + 3^2 + 5^2 + \cdots + (2k-1)^2 + (2(k+1)-1)^2 = P(k) + (2(k+1)-1)^2$$
	We now proceed as follows:
	\begin{align*}
		P(k+1) &= P(k) + (2(k+1)-1)^2 \\
		&= \frac{1}{3}k(4k^2-1) + (2(k+1)-1)^2, \text{ by our inductive hypothesis} \\
		&= \dfrac{1}{3} [4k^3 - k + 3(4k^2 + 4k +1)] \\
		&= \dfrac{1}{3} [4k^3 + 12k^2 + 11k +3] \\
		&= \dfrac{1}{3} (k+1)[4k^2 + 8k + 3] \\
		&= \dfrac{1}{3} (k+1)[4(k^2+2k)+3] \\
		&= \dfrac{1}{3} (k+1)[4(k+1)^2 +3-4], \text{ by completing the square} \\
		P(k+1)&= \dfrac{1}{3} (k+1)(4(k+1)^2 -1) 
	\end{align*}
	So we have now shown that $P(k+1) = \dfrac{1}{3} (k+1)(4(k+1)^2 -1)$ holds. This completes our proof by induction. So it must be true that $1^2+3^2+5^2+\cdots+(2n-1)^2=\frac{1}{3}n(4n^2-1)$ for all $n \in \zpos$.
	
$\hfill \Box$
\newpage

\section*{Problem 2}

Given a planar graph $P=(V,E)$, we have Euler's formula:
$|V|+|F|-|E|=2$, where $F$ is the set of faces of $P$ and $E$ is the
set of edges of $P$.
Let $|V|=n$, where $V$ is the set of vertices of $P$.
Prove that $|F|$ is at most $2n$.
\newline

\noindent
{\bf Proof: }
	We want to show that, for a planar graph $P$ with $n$ vertices, it must be true that $|F| \leq 2n$. \parspace
	If $P$ is a forest or tree, then there is only one face. So $|F| = 1$ and $1 < 2n$ $\forall n \in \zpos$. So $|F| < 2n$ if $P$ is a forest or tree. \parspace
	We now consider all other cases. If you were to count the number of edges from the perspective of each face in $P$, you would reach exactly $2|E|$. Additionally, since we require at least three edges to define a face, it must be true that $2|E| \geq 3|F|$.
	Equivalently, $|E| \geq \dfrac{3}{2} |F|$. Recall that Euler's formula holds for all planar graphs. So we have:
	\begin{align*}
		|V| + |F| - |E| = 2 &\iff n + |F| - |E| = 2 \\
		&\iff n + |F| - \dfrac{3}{2} |F| \geq 2 \\
		&\iff -\dfrac{1}{2} |F| \geq 2 - n \\
		&\iff |F| \leq 2n - 4 \\
		|V| + |F| - |E| = 2 &\iff |F| \leq 2n
	\end{align*}
	Since Euler's formula must be true, it must also be true that $|F| \leq 2n$. \parspace
	So, it has been shown in all cases of planar graphs with $n$ vertices that $|F| \leq 2n$
	
$\hfill \Box$
\newpage

\section*{Problem 3}

Prove that in any simple graph there is a path from any vertex of odd degree
to some other vertex of odd degree.
\newline

\noindent
{\bf Proof: }
	We want to show that any simple graph has the property that there is a path from any vertex of odd degree to some other vertex of odd degree. Recall that a simple graph is a graph where no pair of vertices $a,b$ has more than one edge connecting $a$ and $b$.  \parspace
	We now take a simple graph $G=(V,E)$. There are two cases. Either $G$ is connected or not. A graph that is not connected is the union of connected graphs, so we need only prove the connected case. \parspace
	We now take the connected case of $G$. If $degree(v)$ is even for every vertex $v \in V$, our statement is true since there are no vertices with odd degree. So we need only prove the case where there exists some $u \in V$ such that $degree(v)$ is odd. \parspace
	We now proceed with a proof by contradiction, that is, assume there is no path $p$ connecting vertices $u$ and $w$ where $w \in V$ and $degree(w)$ is odd.
	Recall that we are in the case where $G$ is connected. So for $p$ to not exist, it must also be true that there is no $w \in V$ such that $degree(w)$ is odd. This implies that $u$ is the only vertex in $G$ with odd degree. \parspace
	We can now partition $V$ into two subsets: $\{u\}$ and $C=\{c \in V \mid c \neq u \}$. Note that for $c \in C$, $degree(c)$ must be even. Denote $m$ and $n$ as the sum of the degrees of the vertices in $C$ and $V$ respectively. Since $m$ is the sum of even numbers, $m$ must be even. Now we know that $degree(u)$ is odd. Since $n = degree(u) + m$, $n$ must be odd. Yet, the sum of degrees of a graph is always $2*|E|$, which is even. Since $n$ cannot be both even and odd, a contradiction is found. So it must be true that the path $p$ exists. \parspace
	Since $p$ exists, it must be true that in any simple graph there is a path from any vertex of odd degree to some other vertex of odd degree.

$\hfill \Box$

\section*{Problem 4}

A fully binary tree $T$ is a tree such that all internal nodes have
two children. Prove that a fully binary tree with $n$ internal nodes
in total has $2n+1$ nodes.
\newline

\noindent
{\bf Proof: } 
	Our task is to show that the fully binary tree $T$ with $n$ internal nodes has a total of $2n+1$ nodes. We will define $T_n$ to be the number of nodes in such a tree. So we must show that $T_n = 2n + 1$ for arbitrary $n \in \{0\} \cup \zpos$. We will proceed by induction. \parspace
	Consider the base case of $T_0$, that is, a tree with no internal nodes. Since there are no internal nodes, there is only a root node. So surely, $T_0 = 1$. Now, does $T_n = 2n + 1$ hold? Certainly: $2n+1 = 2*0 +1 = 1$. So the base case holds. \parspace
	We now state the inductive hypothesis. We assume, for some $k \in \{0\} \cup \zpos$, that $T_n = 2n + 1$ holds for all $n \leq k$. \parspace
	We now show that $T_{k+1} = 2(k+1) + 1$. Let $M$ and $N$ be trees with $T_k$ and $T_{k+1}$ internal nodes respectively. By definition of $T_n$, it must be true that $N$ contains one more internal node than $M$, we name this internal node $i$. Since $i$ must have two child nodes, it follows that
	\begin{align*}
		T_{k+1} &= T_k + 2 \\
		&= (2k+1)+2 \text{, by inductive hypothesis} \\
		&= 2k+2+1 \\
		T_{k+1}&= 2(k+1) + 1 
	\end{align*}
	Since $T_{k+1} = 2(k+1) + 1$ holds, our proof by induction is complete. So, it must be true that a fully binary tree $T$ with $n$ internal nodes has a total of $2n+1$ nodes.

$\hfill \Box$
\newpage

\section*{Problem 5}

Given an undirected graph $G=(V,E)$, the breadth-first-search starting at $v\in V$
($bfs(v)$ for short) is to generate a shortest path tree starting at vertex
$v\in V$. The diameter of $G$ is the longest of all shortest paths $\delta(u,v), u,v\in V$.
\newline

When $G$ is a tree, the following algorithm is proposed to compute the
diameter of $G$.
\newline

1. Run $bfs(w), w\in V$, and compute the vertex $x\in V$ furthest from $w$.

2. Run $bfs(x)$ and compute the vertex $y\in V$ furthest from $x$.

3. Return $\delta(x,y)$ as the diameter of $G$.
\newline

Prove that this algorithm is correct; i.e., $\delta(x,y)$ is in fact the
longest among all the shortest paths between $u,v\in V$.
\newline

\noindent
{\bf Proof: }
	Take the end points of the longest shortest path to be $a$ and $b$. We know that $y$ is certainly the farthest vertex from $x$. So whether or not this algorithm is correct relies on whether $bfs(w)$ yields $x = a$. \parspace
	We will prove this by contradiction. That is, we assume there exists some vertex $m \in V$ such that $bfs(m)$ yields a furthest vertex $n$ such that $n \neq a$. \parspace
	Without loss of generality, take $\delta(m,a) \geq \delta(m,b)$. Since $bfs(m)$ yields $n$, it must be true that $\delta(m,n) \geq \delta(m,a)$.
	Recall that $\delta(a,b)$ is the diameter yet we can form a longer diameter by using $n$ and $a$ as endpoints. So we have found a contradiction. This means that $bfs(w)$ must yield the end point $a$ on the longest shortest path. \parspace
	Since an arbitrary point yields an end point of the longest shortest path, it must be the case that the above algorithm computes the diameter.

$\hfill \Box$

\end{document}