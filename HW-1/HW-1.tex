\documentclass[11pt]{article}

\usepackage{../theory}

\begin{document}

\coverpage{1}

\section*{Problem 1}

Prove that $1^2+3^2+5^2+\cdots+(2n-1)^2=\frac{1}{3}n(4n^2-1)$.
\newline

\noindent
{\bf Proof: }
We begin by defining the property 
$P(n) = 1^2 + 3^2 + 5^2 + \cdots + (2n-1)^2 $ 
for any 
$n \in \mathbb{Z}^+$.
Our task is to show that $P(n) = \frac{1}{3}n(4n^2-1)$ for all $n \in \zpos$. We proceed by induction. \parspace
We first must prove $P(n) = \frac{1}{3}n(4n^2-1)$ for the base case of $n = 1$. 
Take the LHS: $P(n) = 1^2 = 1$. 
And now the RHS: 
$ \frac{1}{3}n(4n^2-1) = \frac{1}{3}(1)(4(1)^2-1) = 
\frac{1}{3}(4-1) = \frac{1}{3}(3) = 1$.
Now both the LHS and RHS are 1, and $1 = 1$. Since the LHS is equal to the RHS for $n = 1$, our base case is proved. \parspace
We now state the inductive hypothesis. That is, for some $k \in \zpos$, we assume
$P(k) = \frac{1}{3}k(4k^2-1)$
to be true. \parspace
To complete our proof, we must now show that 
$P(k+1) = \frac{1}{3}(k+1)(4(k+1)^2-1)$
holds. First note that 
$$P(k+1) = 1^2 + 3^2 + 5^2 + \cdots + (2k-1)^2 + (2(k+1)-1)^2 = P(k) + (2(k+1)-1)^2$$
We now proceed as follows:
\begin{align*}
	P(k+1) &= P(k) + (2(k+1)-1)^2 \\
	&= \frac{1}{3}k(4k^2-1) + (2(k+1)-1)^2 &\text{by our inductive hypothesis} \\
	&= \dfrac{1}{3} [k(4k^2 - 1)+ 3(2k+1)^2 ] \\
	&= \dfrac{1}{3} [4k^3 - k + 3(4k^2 + 4k +1)] \\
	&= \dfrac{1}{3} [4k^3 + 12k^2 + 11k +3] \\
	&= \dfrac{1}{3} (k+1)[4k^2 + 8k + 3] \\
	&= \dfrac{1}{3} (k+1)[4(k^2+2k)+3] \\
	&= \dfrac{1}{3} (k+1)[4(k+1)^2 +3-4] &\text{by completing the square} \\
	P(k+1)&= \dfrac{1}{3} (k+1)(4(k+1)^2 -1) 
\end{align*}
So we have now shown that $P(k+1) = \dfrac{1}{3} (k+1)(4(k+1)^2 -1)$ holds. This completes our proof by induction. So it must be true that $1^2+3^2+5^2+\cdots+(2n-1)^2=\frac{1}{3}n(4n^2-1)$ for all $n \in \zpos$.

$\hfill \Box$
\newline

\section*{Problem 2}

Given a planar graph $P=(V,E)$, we have Euler's formula:
$|V|+|F|-|E|=2$, where $F$ is the set of faces of $P$ and $E$ is the
set of edges of $P$.
Let $|V|=n$, where $V$ is the set of vertices of $P$.
Prove that $|F|$ is at most $2n$.
\newline

%\noindent
%{\bf Proof:} ....
%....
%$\hfill \Box$
%\newline

\section*{Problem 3}

Prove that in any simple graph there is a path from any vertex of odd degree
to some other vertex of odd degree.
\newline

%\noindent
%{\bf Proof:} ....
%....
%$\hfill \Box$
%\newline

\section*{Problem 4}

A fully binary tree $T$ is a tree such that all internal nodes have
two children. Prove that a fully binary tree with $n$ internal nodes
in total has $2n+1$ nodes.
\newline

%\noindent
%{\bf Proof:} ....
%....
%draw/include a figure if necessary.
%$\hfill \Box$
%\newline

\section*{Problem 5}

Given an undirected graph $G=(V,E)$, the breadth-first-search starting at $v\in V$
($bfs(v)$ for short) is to generate a shortest path tree starting at vertex
$v\in V$. The diameter of $G$ is the longest of all shortest paths $\delta(u,v), u,v\in V$.
\newline

When $G$ is a tree, the following algorithm is proposed to compute the
diameter of $G$.
\newline

1. Run $bfs(w), w\in V$, and compute the vertex $x\in V$ furthest from $w$.

2. Run $bfs(x)$ and compute the vertex $y\in V$ furthest from $x$.

3. Return $\delta(x,y)$ as the diameter of $G$.
\newline

Prove that this algorithm is correct; i.e., $\delta(x,y)$ is in fact the
longest among all the shortest paths between $u,v\in V$.
\newline

%\noindent

\end{document}