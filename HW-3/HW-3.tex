\documentclass[11pt]{article}

\usepackage{../theory}


\begin{document}
\date{}
%\date{Feb 3, 2020}

\coverpage{3}

\newpage
\section*{Problem 1}

	\noindent
	Design context-free grammars for the following languages
	
	1.1 \indent $A=\{a^nb^m|n\neq 2m\}$.
	
	1.2 \indent $B=\{a^ib^jc^k|i,j,k\geq 0$ and either $i=j$ or $j=k\}$.
	
	1.3 \indent $C=\{a^nb^m|n=3m\}$.
	
	1.4 \indent $D=\{a^nb^m|n\leq m+3\}$.
	\newline
	
	\noindent
	{\bf Proof: } We now give the context grammars for each of the above languages.
	\begin{flalign*}
		1.1 \indent & S \longrightarrow aaSb \mid aB \mid A \mid B & \\
		   	        & A \longrightarrow Aa \mid a & \\
		   	        & B \longrightarrow Bb \mid b & \\\\
		1.2 \indent & S \longrightarrow IC \mid AK & \\
					& I \longrightarrow aIb \mid \epsilon & \\
					& K \longrightarrow bKc \mid \epsilon & \\
					& C \longrightarrow Cc \mid \epsilon & \\
					& A \longrightarrow Aa \mid \epsilon & \\\\
		1.3 \indent & S \longrightarrow \epsilon \mid aaaSb & \\\\
		1.4 \indent & S \longrightarrow X \mid aX \mid aaX \mid aaaX & \\
					& X \longrightarrow aXb \mid B & \\
					& B \longrightarrow Bb \mid \epsilon
	\end{flalign*}
	
	$\hfill \Box$
	\newpage

\section*{Problem 2}

	\noindent
	Decide whether the following grammar is ambiguous.
	\newline
	
	$S\rightarrow AB|aaB$
	
	$A\rightarrow a|Aa$
	
	$B\rightarrow b$ \\
	
	\noindent
	{\bf Proof: } To show that the above grammar is ambiguous, we must show that there exist two leftmost derivations for some string in the grammar. Consider the following two derivations for $aab$. \parspace
	Derivation 1: $ S \derives AB \derives AaB \derives aaB \derives aab$ \newline
	Derivation 2: $ S \derives aaB \derives aab$ \parspace
	The above derivations are both leftmost and distinct, yet they derive the same string. So the given grammar must be ambiguous.
		
	$\hfill \Box$
	\newpage

\section*{Problem 3}

	\noindent
	Convert the following CFG G to an equivalent PDA.
	
	$R\rightarrow XRX|S$
	
	$S\rightarrow aTb|bTa$
	
	$T\rightarrow XTX|X|\epsilon$
	
	$X\rightarrow a|b$
	\newline
	
	\noindent
	{\bf Proof: } We show the equivalent PDA below.
	
	\begin{figure}[h]
		\centering
		\caption{PDA}
		\includegraphics[width=.75\textwidth]{pics/pda.jpg}
		\label{fig:pda}
	\end{figure} 
	
	$\hfill \Box$
	\newpage

\section*{Problem 4}

	\noindent
	Let $G=(V,\Sigma,R,S)$ be the following grammar. $V=\{S,T,U\}$;
	$\Sigma=\{0,\#\}$; and $R$ is the set of rules:
	
	$S\rightarrow TT|U$
	
	$T\rightarrow 0T|T0|\#$
	
	$U\rightarrow 0U00|\#$
	\newline
	
	\noindent
	4.1 \indent Describe $L(G)$ in English. \newline
		
	\noindent
	{\bf Proof: } In English, $L(G)$ consists of the union of two sets. The first is the set of all strings that include two $\#$ symbols with any number of zeros on either side and in between the $\#$ symbols. The second is the set consisting of all strings beginning with $n$ zeros, then a $\#$ symbol, and then $2n$ zeros.
	
	$\hfill \Box$
	
	\noindent
	4.2 \indent Prove that $L(G)$ is not regular.
	\newline
	
	\noindent
	{\bf Proof: } We will prove that $L(G)$ is not regular by contradiction. That is, we will assume that $L(G)$ is regular. Since $L(G)$ is regular, we can apply the pumping lemma for regular languages. Take $p$ to be the pumping length of $L(G)$. Then, for any string $s \in L(G)$ where $|s| \geq p$ (we know $s$ exists since $p$ is finite and we construct an element of $L(G)$ to be as large as needed), the following must be true:
	\begin{enumerate}
		\item $xy^i z \in L(G)$ $\forall i \geq 0$
		\item $|y| > 0$
		\item $|xy| \leq p$
	\end{enumerate}
	We now choose $s = 0^p \# 0^{2p} \in L(G)$. Since $L(G)$ is regular, we can decompose $s=xyz$. Because of condition 3, it must be the case that $xy$ consists of no more than the first $p$ letters of $s$, meaning that $y = 0^a$ where $1 \leq a \leq p$. We then choose $i = 0$ and wonder if it is true that $xz = 0^{p-a} \# 0^{2p} \in L(G)$. \parspace
	This rests on $2 (p - a) = 2p \iff 2p - 2a = 2p \iff -2a = 0 \iff a = 0$, which is certainly not true since $a$ cannot be 0. So $xz \notin L(G)$ and $s$ cannot be pumped for $i = 0$. \parspace	
	Since $s$ cannot by pumped for $i = 0$, the pumping lemma does not apply to $L(G)$. Since the pumping lemma does not apply to $L(G)$, it must be the case that $L(G)$ is non-regular.
	
	$\hfill \Box$
	\newpage

\section*{Problem 5}

\noindent
Convert the following CFG into an equivalent CFG in Chomsky Normal Form

$A\rightarrow BAB|B|\epsilon$

$B\rightarrow 00|\epsilon$
\newline

\noindent
{\bf Proof: } We begin by adding a start rule $ S \longrightarrow A$. This gives 
\begin{flalign*}
\indent & S \longrightarrow A & \\
		& A \longrightarrow BAB \mid B \mid \epsilon & \\
		& B \longrightarrow 00 \mid \epsilon &
\end{flalign*}
We now remove the rule $ B \longrightarrow \epsilon$
\begin{flalign*}
\indent & S \longrightarrow A & \\
& A \longrightarrow BAB \mid BA \mid AB \mid A \mid B \mid \epsilon & \\
& B \longrightarrow 00 &
\end{flalign*}
And now $ A \longrightarrow \epsilon$
\begin{flalign*}
\indent & S \longrightarrow A \mid \epsilon & \\
& A \longrightarrow BAB \mid BB \mid BA \mid AB \mid A \mid B & \\
& B \longrightarrow 00 &
\end{flalign*}
The unit rule $ A \longrightarrow A$ has no consequence so we remove it. In addition, we replace the rule $ S \longrightarrow A$
\begin{flalign*}
\indent & S \longrightarrow BAB \mid BB \mid BA \mid AB \mid B \mid \epsilon & \\
& A \longrightarrow BAB \mid BB \mid BA \mid AB \mid B & \\
& B \longrightarrow 00 &
\end{flalign*}
We now introduce the new rule $ C \longrightarrow AB$
\begin{flalign*}
\indent & S \longrightarrow BC \mid BB \mid BA \mid AB \mid B \mid \epsilon & \\
& A \longrightarrow BC \mid BB \mid BA \mid AB \mid B & \\
& B \longrightarrow 00 & \\
& C \longrightarrow AB & 
\end{flalign*}
Finally, we create the new rule $ D \longrightarrow 0$. This puts us in Chomsky Normal Form.
\begin{flalign*}
\indent & S \longrightarrow BC \mid BB \mid BA \mid AB \mid DD \mid \epsilon & \\
& A \longrightarrow BC \mid BB \mid BA \mid AB \mid DD & \\
& B \longrightarrow DD & \\
& C \longrightarrow AB & \\
& D \longrightarrow 0 &
\end{flalign*}

$\hfill \Box$
\newpage

\section*{Problem 6}

\noindent
Using pumping lemma to prove that the following languages are not
context-free. \\

\noindent
6.1 \indent $L=\{a^nb^jc^k|k=nj\}$.
\newline

\noindent
{\bf Proof: } We will prove that $L$ is not context free by contradiction. That is, we will assume that $L$ is context free. Since $L$ is context free, we can apply the pumping lemma. Take $p$ to be the pumping length of $L$. Then, for any string $s \in L$ where $|s| \geq p$ (we know $s$ exists since $p$ is finite and we could construct an element of $L$ to be as large as needed), the following must be true:
\begin{enumerate}
	\item $uv^i xy^i z \in L$ $\forall i \geq 0$
	\item $|vy| > 0$
	\item $|vxy| \leq p$
\end{enumerate}
Choose $s = a^p b^p c^{p^2} \in L$. We can then decompose $s = uvxyz$. We now consider three cases for $s$. \parspace
The first case is when one of $v, y$ contains more than one type of symbol. Pumping up results in a string not of the form $a^n b^j c^k$, since there will be some symbols in between (ie $aabaabbcccc$). \parspace
We now consider the case where $v = a^{m_1}$ and $y = b^{m_2}$. Pumping up when $i = 2$ results in a string $a^{p+m_1} b^{j+m_2} c^k$. So is $k = (p+m_1)(p+m_2) = p^2 + (m_1 + m_2)p + m_1 *m_2$? No, since $k = p^2 < p^2 + (m_1 + m_2)p + m_1 *m_2$ and one of $m_1, m_2$ is greater than 0. \parspace
Now we have the case $v = b^{m_1}$ and $y = c^{m_2}$. Pumping up for $i = 2$ results in the string $s = a^p b^{p+m_1} c^{p^2 + m_2}$. So is $k = p^2 + m_2 = p(p+m_1) \iff p^2 + m_2 = p^2 + p*m_1 \iff m_2 = p*m_1$. So $s_i \in L$ only when $m_2 = p*m_1$. Can this ever be the case? Well, $m_1 = 0 \implies m_2 = 0$, but that cannot be the case since $|vy| > 0$. So it must be true that $p * m_1 \geq p \implies m_2 \geq p$, yet this also cannot be the case since $|vxy| \leq p$. \parspace
Thus a contradiction is found in all cases and $L$ is not context free.

$\hfill \Box$

\newpage
\noindent
6.2 \indent $L=\{a^n b^j | n \geq (j-1)^3 \}$.
\newline

\noindent
{\bf Proof: } We will prove that $L$ is not context free by contradiction. That is, we will assume that $L$ is context free. Since $L$ is context free, we can apply the pumping lemma. Take $p$ to be the pumping length of $L$. Then, for any string $s \in L$ where $|s| \geq p$ (we know $s$ exists since $p$ is finite and we could construct an element of $L$ to be as large as needed), the following must be true:
\begin{enumerate}
	\item $uv^i xy^i z \in L$ $\forall i \geq 0$
	\item $|vy| > 0$
	\item $|vxy| \leq p$
\end{enumerate}
Choose $s = a^{(p-1)^3} b^p \in L$. Then we can decompose $s = uvxyz$. Now consider the following four cases. \parspace
If one of $v,y$ contains more than one type of symbol then pumping results in leaving the form $a^n b^j$ for any $n,j \in \mathbb{N}$. \parspace
Now consider the case where $vxy = a^{m}$ for $1 \leq m \leq p$. Then we can pump down for the case $i = 0$ so that $s = a^{(p-1)^3 - m} b^p$. Clearly $(p-1)^3 \neq (p-1)^3 - m$ since $m \neq 0$. \parspace
We now view the case where $vxy = b^m$ where $1 \leq m \leq p$. We again pump down ($i = 0$) and now $s = a^{(p-1)^3} b^{p-m}$. Yet it cannot be the case that $(p-1)^3 = (p-m-1)^3$ since $m \neq 0$. \parspace
We now consider the final case where $v = a^{m_1}$ and $y = b^{m_2}$ for both $m_1, m_2 \geq 1$. We then pump up for the case $i = 2$ and now $s = a^{(p-1)^3 + m_1} b^{p + m_2}$. This case rests on the fact that $(p-1)^3 + m_1 = (p + m_2 - 1)^3$. Yet, since $m_2 \geq 1$, it must be true that $(p + m_2 - 1)^3 \geq p^3$. But since $m_1 < p$, it must also be true that $(p-1)^3 + m_1 < (p-1)^3 + p < p^3$. So $(p-1)^3 + m_1 < (p + m_2 - 1)^3$. \parspace
So we have found a contradiction in all cases and $L$ is not context free.

$\hfill \Box$
\newpage

\end{document}
